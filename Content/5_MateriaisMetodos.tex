\chapter{Desenvolvimento do Projeto}
\label{Material}


\par Neste cap�tulo ser�o detalhadas as etapas de desenvolvimento do projeto, bem como as ferramentas utilizadas.
% % % % % % % % % % % % % % % % % % % % % % % % % % % % % % % % % % % % % % % % % % % % % % % % % % %
\section{Material}

\par Para a execu��o do projeto foram necess�rias diversas ferramentas para projetar, desenvolver e testar o sistema. A seguir s�o listados todos os materiais utilizados ao longo do projeto.

\begin{itemize}
\item Para o desenho de diagramas de classe e mapas mentais na fase de projeto, foram utilizadas as ferramentas \href{http://dia-installer.de/}{\textit{Dia}} e \href{https://www.draw.io/}{\textit{Draw.io}}.

\item Foi utilizado o m�todo �gil \textit{Scrum} para a constru��o do sistema. A plataforma \href{https://taiga.io/}{\textit{Taiga}} foi utilizada para organiza��o e planejamento dos \textit{Sprints}.

\item Para controle de vers�o foi utilizado o \href{https://git-scm.com/}{\textit{Git}} sincronizado ao reposit�rio online \href{https://github.com/}{\textit{Github}}.

\item Para o realizar modifica��es no c�digo da IDE do Arduino, foi utilizada a IDE \href{https://www.jetbrains.com/idea/}{\textit{InteliJ IDEA}} para escrita do c�digo e o \href{https://ant.apache.org/}{\textit{Apache Ant}} para a compila��o do c�digo fonte da IDE do Arduino. Tamb�m foi utilizado o \href{https://inkscape.org/pt-br/}{\textit{Inkscape}} para a altera��o no \textit{design} da IDE do Arduino (inser��o do bot�o "Android").

\item Para o desenvolvimento \textit{mobile}, foi utilizada a IDE \href{https://developer.android.com/studio/}{\textit{Android Studio}}.

\item Para criar testes de unidade, foi utilizado o \href{https://junit.org/junit4/}{\textit{JUnit4}} em conjunto com o \href{https://github.com/powermock/powermock}{\textit{PowerMock}} (utilizado como \textit{plugin} do \href{https://gradle.org/}{\textit{Gradle}}).

\item Para montar c�digos \textit{Assembly} escritos para o ATmega328P, foi utilizado \href{http://avra.sourceforge.net/}{\textit{AVRA}}.

\item Foi utilizado um Arduino UNO R3 para comparar os resultados obtidos pelo aplicativo com o sistema real.
\end{itemize}


% % % % % % % % % % % % % % % % % % % % % % % % % % % % % % % % % % % % % % % % % % % % % % % % % % %
\section{M�todo}

\subsection{Desenvolvimento na IDE do Arduino}

\par A primeira parte do desenvolvimento ocorreu na IDE do Arduino. Nela, foi criado um bot�o "Android" cuja fun��o � compilar o c�digo e transferi-lo para o aparelho Android conectado ao computador, da mesma forma que o bot�o "\textit{Upload}" faz com a placa de Arduino.





